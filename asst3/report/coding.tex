\section{LICM: Loop Invariant Code Motion}

\subsection{Dominance Information}

We use the following data flow equations to compute dominance information using the
data flow framework. \\

$\operatorname{Dom}(n_o) = \left \{ n_o \right \}$

$\operatorname{Dom}(n) = \left ( \bigcap_{p \in \text{preds}(n)}^{} \operatorname{Dom}(p) \right ) \bigcup^{} \left \{ n \right \}$,
where $n_o$ is the start node.

\subsection{Finding Invariant Instructions}

We use the following check to determine if an LLVM Instruction pointer I is invariant :\\

\texttt{isSafeToSpeculativelyExecute(I) \&\& !I->mayReadFromMemory() \&\& !isa<LandingPadInst>(I)}\\

This is because TODO

\subsection{LICM Implementation}

TODO

\subsection{Microbenchmarks}

\begin{table}[!ht]
\centering
\begin{tabular}{c|l|l}
  \toprule
  \textbf{Benchmark} & \textbf{Instruction Count} & \textbf{Transformed Instruction Count} \\
  \midrule
  licm-test-1 & 100 & 50 \\ 
  licm-test-2 & 100 & 50 \\
  licm-test-3 & 100 & 50 \\ 
  \bottomrule
\end{tabular}
\caption{Comparison between default bitcode's dynamic instruction count and transformed
  bitcode's dynamic instruction count.}
\end{table}  


\section{Dead Code Elimination}

\subsection{DCE Implementation}

\subsection{Microbenchmarks}

\begin{table}[!ht]
\centering
\begin{tabular}{c|l|l}
  \toprule
  \textbf{Benchmark} & \textbf{Instruction Count} & \textbf{Transformed Instruction Count} \\
  \midrule
  dce-test-1 & 100 & 50 \\ 
  dce-test-2 & 100 & 50 \\
  dce-test-3 & 100 & 50 \\ 
  \bottomrule
\end{tabular}
\caption{Comparison between default bitcode's dynamic instruction count and transformed
  bitcode's dynamic instruction count.}
\end{table}  

\newpage

%%% Local Variables:
%%% mode: latex
%%% TeX-master: "asst1"
%%% End:
